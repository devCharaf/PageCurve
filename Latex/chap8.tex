In this note we started with a general system with 2 CFTs separated with an interface. This system extends to gravitational solutions in the bulk separated by a domain wall. We were interested in the case where one slice has larger degrees of freedom to model something close to a black hole following the central Dogma. In this limit (as well as large boundary for the other slice), we found that the dominant phase is the hot phase with a part of the horizon on each side. The geometry was computed explicitly in the case of $[\text{H}2,\text{H}2]$. 

We then computed RT surfaces in the single sided geometry. This problem turned out to be computations of space-like geodesics as the metric is time independant. As we take the limit of large boundary in the red slice, the true RT surface is the one that crosses the wall. The latter is $\sigma$ dependant geodesic. The large $\ell_2$ limit considered implies that the true geodesic is that close to the horizon, $\sigma=0$.

We finally discussed a version of the Hawking information paradox involving a two-sided hot phase staring in the Hartle-Hawking state, and evolved forward in time on both sides. We computed RT surfaces homologous to the boundary subregion $\Tilde{A}$. One geodesic was found to be a increasing function of time, which is a version of Hawking's information paradox. A second RT surface was found to be constant over time, and therefore takes over the first entropy as we take the minimum of the two. This RT surface is complicated to compute in the double sided geometry as it can move in the $q$ direction as well. However, it can be computed in the original one sided geometry since it consists of two parts, each remaining in its corresponding side.

This study is similar to the work presented in \cite{almheiri2019islands}. Our system is $2+1$ dimensional and spiced with a wall separating two geometries. However, the result of the final entropy in this example stays consistent with the literature discussed above.


\textbf{Acknowledgment}

I would like to thank Prof. João Penedones for his assistance and his helpful advice.
I would also like to thank Kelian Häring for allowing me to push forward
during this Thesis.