This Appendix refers to chapter \ref{section 2}. We detail some computations that we believe to be important for a complete report.

\section{Bell state}

Consider the Bell state (\ref{Bell pair}). The reduced density matrix matrix was found by tracing over $B$ and is given in (\ref{psiA}). From $\rho_A$ we find,
\begin{equation}
    \rho\ln\rho= \frac{\ln(2)}{2}\left(\ket{0}_A\bra{0}+\ket{1}_A\bra{1}\right).
\end{equation}
Plugging this in the definition of von Neumann entropy, we find
\begin{equation}
    S = \ln(2).
\end{equation}
This shows that $A$ and $B$ are maximally entangled.

\section{Strong Subadditivity}
The strong subadditivity condition (\ref{weak monotonicity}) can be rewritten by mean of the conditional entropy,
\begin{align}
    S\left(AB\right) + S\left(BC\right) \geq S\left(A\right) + S\left(C\right) &\implies S\left(AB\right) -  S\left(A\right)  + S\left(BC\right)-S\left(C\right)\geq 0\\
    &\implies H\left(B|A\right) + H\left(B|C\right)\geq 0.
\end{align}

\section{thermofield double}

In this appendix we check using path integrals that tracing over the system $B$ in equation \ref{thermofield double} gives back the original Gibbs state. 
\begin{align}
    _A\bra{x_0}\text{Tr}_B\left(\rho=\ket{\phi}\bra{\phi}\right)\ket{x_1}_A &= \int\text{d}y\bra{x_0,y}\ket{\phi}\bra{\phi}\ket{x_1,y}\\
    &= \frac{1}{Z}\int\text{d}y ~~ \ctikz{\draw (-1,-0.15) arc (190:350:1);\node at (-1,-0.15) [circle,fill=red,inner sep=1.5pt]{}; \node at (1,-0.15) [circle,fill=red,inner sep=1.5pt]{};
    \draw node at (-1.3,-0.15){$y$};\draw node at (1.3,-0.15){$x_0$};\draw node at (-1.3,0.15){$y$};\draw node at (1.3,0.15){$x_1$}; \draw (1,0.15) arc (10:170:1);\node at (-1,0.15) [circle,fill=red,inner sep=1.5pt]{}; \node at (1,0.15) [circle,fill=red,inner sep=1.5pt]{};}\\
    &= \frac{1}{Z} \cdot \ctikz{\draw (0.967,0.255) arc (10:350:1);\node at (0.967,0.255) [circle,fill=red,inner sep=1.5pt]{}; \node at (0.96,-0.150) [circle,fill=red,inner sep=1.5pt]{};
    \draw node at (1.3,-0.2){$x_0$};\draw node at (1.3,0.2){$x_1$};},
\end{align}
which is exactly the Gibbs state from the definition \ref{l3iba}.