This Appendix refers to chapter \ref{section 3}.

\section{Geodesics in AdS space time}

In this section, we compute space like geodesics inside AdS space time in 2+1 dimension.The geometry is explicitly described in equation \ref{AdS_3 metric}. In static geometries we are interested in space like geodesics to compute RT surfaces. The metric of interest becomes,
\begin{equation}\label{static ads}
    \text{d}s^2= \frac{\ell^2}{z^2}\left(\text{d}x^2+\text{d}z^2\right).
\end{equation}

Geodesics can be found by solving Einstein field equations or simply by treating the metric \ref{static ads} as a one dimensional Lagrangian,
\begin{equation}\label{Lagr}
    L = \int\frac{\text{d}x}{z}\sqrt{\dot z^2+1}.
\end{equation}
The dot correspond to a derivative with respect to $x$. As the Lagrangian is time independent, the energy conservation reads,
\begin{equation}
    E = \frac{1}{z\sqrt{\dot z^2+1}}.
\end{equation}
The solutions of this differential equation is,
\begin{equation}
    z = \pm \sqrt{E^{-2}-(x-x_0)^2},
\end{equation}
where $x_0$ is a constant of integration. This can be rearranged in as written in eq. (\ref{AdS_3 vaccum geodesic}) with $r=\frac{1}{E}$.

Plugging this solution back in eq. (\ref{Lagr}), we find the length of the geodesic,
\begin{align}
    L &= 2r\ell \int_0^\frac{L}{2} \text{d}x ~ \frac{1}{z^2}\\
    &= 2r \int_0^\frac{L}{2} \text{d}x ~ \frac{1}{(x-x_0)^2-r^2}\\
    &= \log\left(\frac{r+x-x_0}{r-x+x_0}\right)\Big|^\frac{L}{2}.
\end{align}
Replacing with $x_0 = 0$ and $r^2 = \left(\frac{L}{2}\right)^2+\epsilon^2$ we get exactly eq. (\ref{first RT}).

\section{Hawking temperature}

Consider the metric (\ref{BTZ}) with the change of coordinate $z=1/r$,
\begin{equation}
    \text{d}s^2 = \ell^2\left(-g\left(r\right)\text{d}t^2+\text{d}x^2+\frac{\text{d}r^2}{g\left(r\right)}\right),
\end{equation}
where $g(r) = r^2-r_h^2$ with $r_h=1/z_h$.

By making a Wick rotation $\tau=it$ we get the Euclidean geometry,
\begin{equation}
    \text{d}s^2 = \ell^2\left(g\left(r\right)\text{d}\tau^2+\text{d}x^2+\frac{\text{d}r^2}{g\left(r\right)}\right).
\end{equation}
Near the horizon we can rewrite the metric by expanding $g(r)$ around $r_h$. We make the following change of coordinate,
\begin{align}
    (r-r_h)=\frac{1}{4}g'(r_h)\rho^2, && \tau = \frac{\beta}{2\pi}\theta,
\end{align}
where $\beta$ is the period of $\tau$. The metric becomes,
\begin{equation}
    \text{d}s^2 = \ell^2\left(\left(\frac{g'(z_h)\beta}{4\pi}\right)^2\text{d}\tau^2+\text{d}x^2+\text{d}\rho^2\right).
\end{equation}

In order to avoid the conical singularity at $r=r_h$, we need to take the factor multiplying $\text{d}\theta$ to be 1. The point at the horizon shouldn't be something special compared to the rest of space-time outside the horizon.
\begin{equation}
    \frac{g'(z_h)\beta}{4\pi} = 1\implies T = \frac{r_h}{2\pi} = \frac{1}{2\pi},
\end{equation}
where $\beta = \frac{1}{T}$.