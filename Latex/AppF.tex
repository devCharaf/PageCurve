This Appendix refers to chapter \ref{section 7}.

\section{Cutoff surface}

The cutoff surface in the $(q,p)$ coordinates is not a trivial surface at constant $p$. It is rather a $(q,p)$ dependant surface. To find this surface we have to map $r$ to $(q,p)$ through (\ref{Kruskal coord}) and (\ref{Penrose coord}),
\begin{equation}
    r = r_h\frac{\cos(q)}{\cos(p)}.
\end{equation}
Then the cutoff surface is given by,
\begin{equation}\label{cutoff}
    \frac{1}{\epsilon} = r_h\frac{\cos(q)}{\cos(p)}\implies \cos(p)=\epsilon r_h \cos(q).
\end{equation}

\section{$q$ to $t$ at the cutoff}

We want to recover the time dependence of the RT surface in eq. (\ref{c'est bon}). Using (\ref{Kruskal coord}) we find,
\begin{equation}
    \frac{v}{u} = \tanh\left(\frac{r_h t}{\ell}\right).
\end{equation}

And from (\ref{Penrose coord}), we find,
\begin{align}
    \frac{v}{u} &= \frac{\sin(q)}{\sin(p)}\\
    &= \sqrt{\frac{1-\cos^2(q)}{1-\epsilon^2r_h^2\cos^2(q)}},
\end{align}
where we used (\ref{cutoff}).

Rearranging these two equations gives an expression for $\cos(q)$
\begin{equation}
    \cos^2(q)=\frac{1-\tanh^2(2\pi t/\beta)}{1-\epsilon^2{r_h}^2\tanh^2(2\pi t/\beta)},
\end{equation}
where we used $\frac{r_h}{\ell}=\frac{2\pi}{\beta}$.
